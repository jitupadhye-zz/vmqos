\section{Implementation}
\label{sec:implementation}
In this section, we briefly discuss the implementation of key components of MBFQ in Windows
Server 2016.

{\bf Binary location}: MBFQ is implemented in Windows as a Hyper-V Virtual
Switch Extension, which is an NDIS light-weight filter (LWF) driver~\cite{fd}.

Like other LWF drivers, MBFQ uses a standard NDIS API to receive outgoing
packets from the upper stack and sends conformed packets to the lower stack
through another standard NDIS API.  MBFQ implementation is completely abstracted
from the application above and the hardware below.  The only information from
hardware that MBFQ receives is from a standard NDIS API for receiving link state
status (NDIS\_STATUS\_LINK\_STATE) to determine the link capacity of the underlying 
network adapter.

{\bf The macroscheduler}: MBFQ uses a timer to invoke the macroscheduler every
100 milliseconds. if the network adapter utilization is below 80\% the
macroscheduler deactivates the microschedulers (if they have not already been
deactivated), and immediately returns.  Thus, when the link utilization is low,
which is the common case in data centers~\cite{fb,kandula09}, outgoing traffic
is not throttled, and MBFQ uses minimal CPU.  If the link utilization is above
80\%, the macroscheduler computes and distributes the transmit rates to the
microschedulers as discussed earlier.

{\bf The microschedulers}: The microschedulers are implemented as token buckets,
whose rates are periodically adjusted by the macroscheduler.  Each
microscheduler can schedule packets across multiple processors, using
per-processor sub-queues that share a common token bucket.  This provides a
fully distributed weighted fair queuing implementation where packets can remain
on the same processor from the application layer to the NIC's transmit buffer.


