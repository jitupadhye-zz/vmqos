\begin{abstract}
We wish to allocate outgoing bandwidth at a server among customer VMs. The
allocation for each VM is proportional to the bandwidth purchased for that VM by
the customer, and any idle bandwidth is also proportionally redistributed. This
is the classical fair queuing problem. However, most solutions~\cite{drr,qfq,
wf2q} to the classical fair queuing problem assume tight feedback between
transmitter and scheduler, and cheap scheduler invocation on every transmission.
Since these assumptions are false in Virtual Switches, we propose MBFQ
(Measurement Based Fair Queuing) with two levels of scheduling: a microscheduler
that operates cheaply and paces VM transmissions, and a macroscheduler that
periodically redistributes tokens to microschedulers based on the measured
bandwidth of VMs. We show that MBFQ allows a VM to obtain its allocated
bandwidth in three scheduling intervals, and that idle bandwidth is reclaimed
within five periods. An implementation of MBFQ  is available in Windows Server
2016 Technical Preview.
\end{abstract}
