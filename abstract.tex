\begin{abstract}
We wish to allocate outgoing bandwidth at a server among customer VMs. The
allocation for each VM is proportional to the bandwidth purchased for that VM by
the customer, while allowing idle bandwidth to redistributed.  Classical fair
queuing in routers assumes tight feedback between transmitter and scheduler, and
that cheap scheduler invocation on every transmission. Since these assumptions
are false in Virtual Switches, we propose MBFQ (Measurement Based Fair Queuing)
with two levels of scheduling: a microscheduler that operates cheaply and paces
VM transmissions, and a macroscheduler that periodically redistributes tokens to
microschedulers based on the measured bandwidth of VMs. We show that MBFQ allows
a VM to obtain its allocated bandwidth in X scheduling intervals, and that idle
bandwidth is reclaimed within Y periods. The MBFQ code previewed in Windows
Server in Jan 2015.
\end{abstract}
