\begin{abstract}
We wish to allocate bandwidth at a server among customer
VMs proportional to the bandwidth purchased by each customer
while allowing idle bandwidth to redistributed. This
specification is classic fair queuing, with a 30-year old research
literature. However, classical fair queuing assumes a
tight feedback loop between the transmitter and the scheduler,
and also assumes that the scheduler can be cheaply invoked
on every transmission. While true in routers, these
assumptions are false in Virtual Switches with a software
scheduler and a hardware NIC. Instead, we propose MBFQ
(Measurement Based Fair Queuing) with two levels of scheduling:
a microscheduler that operates cheaply and paces VM
transmissions, and a macroscheduler that periodically redistributes
tokens to microschedulers based on the measured
bandwidth of VMs. We show using theory and experiments
that MBFQ allows a VM to obtain its allocated bandwidth in
X scheduling intervals, and that idle bandwidth is reclaimed
within Y periods. The MBFQ code will preview in theWindows
Server QoS code in Jan 2015.
\end{abstract}
