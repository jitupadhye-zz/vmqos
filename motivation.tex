\section{The Case for WFQ with more relaxed constraints}

Many applications, such as file server, web server, video streaming, do not need
per-packet level fairness.  These applications are already built to handle
network congestions, delays and jitters, and continually pushes for as much
bandwidth (e.g. via TCP windows) as needed to meet their demands.  These
applications can tolerate some small ramp-up delay in getting their bandwidth
reservation.  Our experiments show that a 300ms ramp-up delay do not have any
noticeable effect on these applications.  Furthermore, our experiments show that
these applications are more susceptible to fluctuations in bandwidth allocations
than they are to slow ramp-up time. The requirements to support bandwidth
reservation for these applications are:

\begin{itemize}
\item
The queue needs to be allocated its bandwidth reservation within a time that is
significantly shorter than the duration of the application.  Given that these
applications typically send traffic on the order of minutes, a 300ms ramp-up
delay would be sufficiently short.  
\item The allocated bandwidth should not
fluctuate faster than the ramp-up delay.
\end{itemize}

Given these requirements, a lot of the complexity and cost that comes with
ensuring per-packet level fairness can be eliminated, making it simple to
implement a bandwidth reservation framework that is hardware-agnostic, and works
in a general-purpose host. 

Our solution forgoes per-packet level fairness WFQ.  Instead, we use a
measurement-based WFQ approach with a more relaxed constraints on fairness and
bandwidth guarantee.  Our solution substantially reduces the complexity of
per-packet level WFQ and can scale across multiple CPUs, while satisfying the
following SLA:

Given a set of flows $F=[f1, f2, \ldots fn]$ with demands $D=[d1,d2,\ldots,dn]$
and weights $W=[w1,w2,\ldots, wn]$, sharing an output link of capacity $C$, a
set of bandwidth allocations $A=[a1, a2, \ldots, an]$ is enforced such that:
$(i)$Sum(A) <= C $(ii)$	If $ai < C*wi$, and $di > ai$, then $ai=min(di, R*wi)$ within
a time period T.  $(iii)$ When the system is in steady state, if $Sum(D) >= C$,
then $Sum(A) = C$.

The first condition provides flow isolation on the shared resource to guarantee
each flow its bandwidth allocation. The second condition guarantees that a
flow�s bandwidth allocation will satisfy its demand, up to its reservation,
within a time T.  Our current implementation guarantees T to be 300ms or less.
The third condition provides the work-preserving property of WFQ in
steady-state.  This distinguishes our implementation from a trivial bandwidth
guarantee implementation with static multiplexing (i.e. providing a static max
cap for each VM)

We implemented our WFQ scheduler as an NDIS filter driver in Windows.  As
packets are sent from the application layer, through the NDIS stack, on multiple
processors, the filter driver enforces bandwidth reservation by shaping the
per-processor queues, and therefore avoids the need to re-queue packets on a
different processor.  The advantages of having packets processed on the same
processor throughout the entire stack include: Avoiding L1 \& L2 cache misses,
and preserves the distributed nature of packet processing in modern operating
systems.  Eliminating cross-processor lock contention in the data path.
Furthermore, our filter takes less than 5\% of any CPU to enforce bandwidth
reservation when the network interface is congested.  When the network interface
is not congested, our filter is just a pass-through, and is virtually out of the
data path.

