\section{Introduction}

Network bandwidth reservation is a feature that guarantees a certain amount of
network bandwidth for a flow that uses a shared resource, typically a network
adapter.  When the flow has demand, it is guaranteed at least up to its
bandwidth reservation value.  When the flow transmits below its reservation, or
is idle, the bandwidth becomes available for other flows to consume.  

Bandwidth reservation is generally achieved with Weighted Fair Queuing, which
isolates flows into queues of different weights, and allocates bandwidth to each
queue according to its demand and relative weight.  WFQ generally implies a
packet-level fairness across queues.  In packet-based switches and routers, WFQ
is implemented with a round-robin scheduler that schedules packets out of each
queue according to their weights and demand.  Implementations of WFQ in this
space are well-understood, backed up by decades of research.  These
implementations are often done either in hardware, or in special-purpose
software with direct access to the hardware API (i.e. device driver or
firmware).  As we move the implementation of WFQ further away from the hardware
and make the software more general-purpose, it becomes more complex and costly
to enforce packet-level fairness at the hardware.

As we move the implementation of WFQ further away from the hardware and make the
software more general-purpose, it becomes more complex and costly to enforce
packet-level fairness at the hardware.

\begin{itemize}
\item
A round-robin packet scheduler across all queues needs to run on a single
processor in order to prevent out-of-order packet delivery among packets within
each flow.  In order to keep up with data-center-level packet rates, such
single-core scheduler puts limitations on how much packet processing can be done
between the scheduler and the hardware.
\item
On the other hand, having per-queue or per-processor schedulers requires tight
coordination among schedulers, at packet-level granularities, which is not
feasible at very high data rate (40Gbps -- 100Gbps).  
\item
Furthermore, WFQ requires a feedback loop between the schedulers and the
hardware transmit buffer, which might not exist when the software module is
sufficiently removed from the hardware API.
\end{itemize}

There are, however, many benefits to being able to enforce bandwidth reservation
in software, in a generic operating system, without any special hardware
dependency.  It allows the creation of a bandwidth reservation framework on top
of any generic network interface, which eliminates the need for expensive
hardware.  The benefits are even greater if such software can operate without
consuming excessive cycles from any one processor, or is limited to a single
processor, as it would enable bandwidth reservation without requiring a
dedicated machine, or a dedicated processor, and be able to scale out when more
CPUs are available.

In this paper, we present a WFQ solution that satisfies the following
constraints: (i) The scheduler must run on top of a general-purpose software
stack, communicating with the NIC hardware via general hardware abstraction
layer provided by a general OS. (ii) Packet processing needs to scale across
multiple CPUs.  This constraint is necessary in order to avoid the potential of
having WFQ enforcement be the network bandwidth bottleneck.  It allows the
scheduler to scale with more CPUs.
