\section {Introduction}

In this paper, we revisit the age-old problem of weighted fair
allocation of bandwidth among competing ``flows''. Why do we need a new solution
for such a well-studied problem? The reason is that the new context and
requirements caused by {\em software implementation} in a{\em virtualized cloud}
both demand and allow a simple and CPU-efficient solution. 

Fair bandwidth allocation has typically been studied in the context of a router.
It was assumed that one had to deal with thousands, if not
millions of flows. It was also assumed (perhaps unjustifiably) that
bandwidth had to be apportioned at a fine granularity approaching ideal
processor sharing. A tight coupling between the packet transmission engine and
the packet scheduler was also assumed, since both were implemented in switch
hardware.  For example, the DRR~\cite{drr} implementation assumes that the
scheduler is woken up on every packet departure. Further, the "CPU utilization"
was important only in the sense that the switch hardware had to capable of
handling the expected workload because the CPU (switch hardware) was dedicated
to scheduling.

Our context is quite different. We want to build a packet scheduler for Windows
hypervisor that ensures that VMs hosted on a single server share outgoing
bandwidth in proportion to specified weights. Customized versions of this
Vswitch powers Azure, and many large private corporate clouds. Our context
implies:

\noindent {\bf No hardware assumptions:} We must support legacy deployments, and so cannot assume NIC hardware support for
packet scheduling which must, instead, be done in the CPU.

\noindent {\bf High coordination costs:} Coordination between a
software Vswitch and a hardware NIC is very expensive at Gigabit speeds, requiring optimizations
like Large Send Offload.  Transmission from a VM to the NIC
should ideally bypass a coordinating scheduler to
minimize context switches, and avoid access to shared state to minimize lock
overhead.  Further, to scale to 40 Gbps, the scheduler should be distributed
across cores instead of being single-threaded. 

\noindent {\bf CPU cycles are precious:} For a cloud provider, any CPU cycles saved from
say packet scheduling can be resold for profit, especially with the advent of
elastic computing~\cite{aws}.  This requires us to depart from the state-of-the
art software solutions such as QFQ~\cite{qfq} that requires a CPU to do
significant processing {\em on every sent packet}.

\noindent {\bf Fine-grained guarantees unnecessary:} Public and private cloud providers typically host 
less than 100 VMs per physical server, and provide only coarse per-VM bandwidth guarantees
determined by the pricing tier.  A granularity of 1 Gbps typically suffices.  Fine-grained, per-flow bandwidth is neither
used nor demanded by the customers, since application-layer issues often have
far more impact on throughput.

We {\em do}, however, want our scheduler to be as
work-conserving as possible, and we {\em do} want to allocate any spare
bandwidth roughly proportionally among backlogged VMs.

We designed a packet scheduler to provide roughly proportional bandwidth sharing
with  minimum CPU overhead by refactoring scheduling into two parts: a {\em
microscheduler} that controls the send rate of a VM using a leaky bucket,
and a {\em macroscheduler} that periodically allocates tokens to each
microscheduler based on VM bandwidth usage.  Coordination costs are small and
limited to the macroscheduler that runs once every scheduling period $T$ (say 10
msec); by contrast, each microscheduler can run in a separate thread/core
allowing scaling to 40 Gbps of overall transmit bandwidth with little CPU
overhead, as we demonstrate in Section~\ref{sec:experiments}.

Our design tradeoff is that when demand changes, it takes a few 
macroscheduler periods ($T$) before the system converges to fair allocations,
whether in allowing a previously idle VM to recapture its allocated bandwidth or
to redistribute bandwidth when a VM's demand falls.  $T$  must be high not just
to reduce CPU overhead but also to stably measure the bandwidth demand of a VM
in the face of bursts.

Since the NIC does not provide per-packet feedback, the macroscheduler also
needs to solve a {\em congestion problem} to ensure that the bandwidths
allocated do not exceed the NIC bandwidth.  Thus, reminiscent of TCP~\cite{tcp},
the macroscheduler gradually increases/decreases the allocated bandwidth to a VM
based on its measured demand in the last period.  

Unlike TCP, however, the VM clients share common state that can be used to
enable eventual {\em weighted fairness}, which is not a goal of TCP.  Estimating
future bandwidth by measurement is also similar to  Measurement Based Admission
control (MBAC)~\cite{mbac}.  However, MBAC is about {\em admission control} of
flows  and not on real-time {\em fair queing} of packets. 

Section~\ref{sec:modell} describes our model and Section~\ref{sec: measurements}describes
some motivating measurements.  Section~\ref{sec:algorithm}
describes our MBFQ algorithm, while Section~\ref{sec:implementation} describes
our implementation in Windows server.
We provide simple theoretical bounds fpr MBFQ in
Section~\ref{sec:theory} and describe an experimental evaluation 
in Section~\ref{sec:experiments}.  We survey related work in Section~\ref{sec:relatedwork} followed by a
discussion in Section~\ref{sec:discussion}.
