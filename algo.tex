\section{MBFQ Algorithm}
\label{sec:algorithm}

\begin{figure}
{\footnotesize
\begin{algorithmic}
\State //
\State // MG: minimim guaranteed rate (what the customer has paid for)
\State // SR: measured send rate since last macro scheduler iteration.
\State // TR: computed target rate for this iteration.
\State // AR: currently allocated rate
\State //
\State $residualBandwidth = lineRate$
\State $totalNeedsMoreWeight = 0$
\end{algorithmic}
}
\caption{Notation and Initialization of MBFQ}
\end{figure}

The main idea of MBFQ is to work in intervals of length $T$ where $T$ is
sufficiently large to reduce variance in measurement of sending rates for
typical network applications, and yet not too much larger than is strictly
necessary because the larger $T$ is, the larger $T_{inc}$ and $T_{dec}$ is.
This is because the macroscheduler only adjusts allocated rates in intervals of
length $T$.   We will show in the experimental section that if $T$ is around
$100$ msec, then the standard deviation is around $5\%$ for a TCP sender; while
the deviation goes down to $1\%$ as $T$ increases to $1$ seconds, it also slows
down responsiveness.  Hence our code uses $T = 100$ msec as a default.

The basic idea is simple.  We measure what every VM has sent in the last $T$
seconds and store in a variable $SR$ (for send rate).  In the first phase, by
comparing the sent rate to the amount  allocated to that VM in the last interval
($AR$), we make an estimate of whether this VM's allocation should (in an ideal
world with no constraints) increase or decrease and compute a target rate $TR$
in this ideal world. Then in a second phase, we incorporate the reality of a
fixed pot of link bandwidth and the other VMs that wish to send in order to
enforce fair-queuing and congestion freedom.

{\bf Phase 1: Determining Ideal Rates:}  First, we determine ideal rates for
each VM in an world where this VM is the only player based on past performance.
For example, if $SR > 0.95 AR$, then there is a case to be made that the VM
should be allocated more bandwidth (it has used what it was allocated).  Asking
a VM to use up everything it was allocated seems too strict, becausae of natural
variations in sending rates.  Of course, $95\%$ is merely a parameter.    

Similarly, if a VM's sending rate has fallen by some percentage of its allocated
rate, we might want to reduce the VM' s allocated rate.  Here we might choose to
be cautious.  A VM that has paid for $4$ Gbps. and was allocated that much, may
be measured to fall briefly below $4$ for a few intervals.  While if this
situation persists, we should reallocate the bandwidth, we wish to guard against
short term fluctuations that cause the algorithm to too eagerly reclaim
bandwidth.  Since it takes a few intervals to rise again, we choose to defend
against this by rising fairly quickly (say in 4 periods), but reclaiming
bandwidth at a much slower rate (say 10 periods).   

In summary, the first phase of the algorithm decides a target rate $TR$ based on
past sending performance.  If the VM has used its allocation, we mark it for
increase; otherwise, we keep it the same or even decrease it (after some number
of periods demonstrating this behavior) to close to its sending rate.    This
brings up the question: how should we increase a VM (still in an ideal world, we
will consider sharing in the second phase).  

To see the difficulty, suppose a VM $A$ has paid $2000$ a month for $5$ Gbps of
traffic and has been idle for a long time, and hence was allocated only a small
minimal amount of bandwidth of say $10$ Mbps.  Other VMs are using portions of
$A$'s bandwidth.  $A$ now decides to initiate a large data transfer which
requires its full guaranteed rate.   But $A$ has been allocated only $10$ Mbps.
We wish $A$ to rise rapidly from $10$ Mbps to $5$ Gbps.

In the first interval after $A$ initiates the data transfer $A$ will be measured
to have sent $10$ Mbs.  Since $A$ has shown intent to send, why not increase $A$
directly to $5$ Gbps.   The difficulty is if $A$ for other reasons can only send
at say $1$ Gbps.  In that case raising $A$ to $5$ can "waste" 4 Gbps for one
interval.  So a more cautious strategy seems to be in order.

Instead, we keep some state and (almost like TCP) and gradually increase $A$'s
allocations, taking care to bound $A$'s rise to its guaranteed rate.   For
example, in our current code, in the first period, we give $A$, $20\%$ more
allocation; if $A$ uses that up, we give $A$, $50\%$ more bandwidth.  However,
after some number of periods in which $A$ uses up all its allocations, we simply
aim to give $A$ at least its guaranteed bandwidth.   We bound that by 4 periods
in our current code.    

{\bf Phase 2:  Preventing Congestion and Enforcing Fair Sharing:}  So far the
algorithm has only determined the ideal allocations for each VM based on their
traffic pattern displayed in the last period.   However, to achieve
congestion-freedom, statistical multiplexing and fair sharing we must do more.
If the sums of the ideal rates computed in the first phase exceeds the lnk
bandwidth, we must prevent congestion.  If some VMs's ideal rates are below
their guaranteed rates, we must distribute unused bandwidth in proportion to the
weights of all flows that wish to use this bandwidth.  

Phase 2 begins by initializing each VM's allocated rate to be the minimum of its
ideal rate and its guaranteed rates.  This initial allocation guarantees
congestion freedom since by definition, the sum of the guaranteed rates cannot
exceed the link bandwidth.  But it can leave bandwidth on the table. This
remaining bandwidth is the link bandwidth minus the sum of the initial
(conservative) allocations.

To distribute "remaining" bandwidth, the algorithm marks the VMs whose ideal
target rates are more than their guaranteed rates as VMs that "need more
bandwidth".  Phase 2 then goes through a loop where it allocates the remaining
bandwidth to the needy VMs in proportion to their weights.  

This is subtler than it may appear.   At the end of one sub-iteration of this
second allocation phase, one may have allocated a VM more than its ideal rate
calculated in the first phase.  Since, the ideal rate is our proxy for what the
VM would desire in an ideal world without sharing, we remove this bandwidth and
iterate again.  

Once again, we mark VMs whose allocated rates are still less than their ideal
rates as needy, and allocate the remaining bandwidth among the still needy VMs.
This iteration continues until no bandwidth is left or no VMs are needy.   We
can prove that this loop terminates in that at least one VM will be removed from
the needy list in each iteration.  Note that these iterations within Phase 2 are
merely a few lines of code that run in nanoseconds (not in intervals of $T$); so
even doing this for 100s of VMs will not take be too slow.

The initialization for the MBFQ algorithm is shown in
Figure~\ref{initialization}.  Note that $RUStage$ (for Ramp Up Stage) is a state
variable that tracks the number of iterations over which a VM has been
increasing.  Every time a VM comes close to it's target, this counter is
increased (but to no more than $3$); every time a VM falls significantly below
its target this counter is decreased.  The 

\subsection{PseudoCode}

The pseudocode for the first part of Phase 1 is shown in
Figure~\ref{phase1start}. This code determines the slope of the VM's bandwidth
change (needs to increase, decrease or stay the same) by comparing the amount
allocated in the last period to what the VM sent, using some percentage
thresholds for some margin.  If it is falling, $RUStage$ is decreased; if it is
increasing $RUStage$ is incremented, always being careful to keep $0 \leq
RUStage \leq 3$. 

The pseudocode for the second part of Phase 1 is shown in
Figure~\ref{phase1finish}.  This is where the number of consecutive iterations
over which a VM has been determined to increase (tracked by $RUStage$) is used
to determine how aggressive an ideal increase to allocate to the VM, starting
from a modest $10\%$ increase all the way to at least the guaranteed bandwidth
in 4 iterations.

The pseudocode for the first part of Phase 2 is shown in
Figure~\ref{phase2start}.  It uses a new variable called $TempRate$ that is
initialized to be no more than the guaranteed rate.   This is then used to
compute how much bandwidth is left over, and which VMs can (based on their ideal
allocations) use up this remaining bandwidth.

The pseudocode for the second part of Phase 2 is shown in
Figure~\ref{phase2finish}.   It is a loop that allocates the excess to all the
needy VMs. If each VM then remains below its ideal target rate, the loop
terminates.  However, if some VM is allocated more than it uses, this wasted
bandwidth is added back to the remaining bandwidth and the loop continues.


